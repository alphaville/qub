\documentclass[a4paper,10pt,reqno]{amsart}

\usepackage[utf8]{inputenc}
\usepackage[foot]{amsaddr}
\usepackage{amsmath,amsfonts,amssymb,amsthm,mathrsfs}
\usepackage[margin=0.95in]{geometry}
\usepackage{color}
\usepackage{units}
\usepackage[dvipsnames]{xcolor}


\usepackage{mathtools}
\usepackage{enumerate}
\usepackage{mathrsfs} 			% for mathscr
\usepackage{hyperref}

\definecolor{CommentGreen}{rgb}{0.0,0.4,0.0}
\definecolor{Background}{rgb}{0.9,1.0,0.85}
\definecolor{lrow}{rgb}{0.914,0.918,0.922}
\definecolor{drow}{rgb}{0.725,0.745,0.769}


\numberwithin{equation}{section}
\synctex=1

\hypersetup{
    bookmarks=true,         % show bookmarks bar?
    unicode=false,          % non-Latin characters in Acrobat’s bookmarks
    pdftoolbar=true,        % show Acrobat’s toolbar?
    pdfmenubar=true,        % show Acrobat’s menu?
    pdffitwindow=false,     % window fit to page when opened
    pdfstartview={FitH},    % fits the width of the page to the window
    pdftitle={PID Quiz},    % title
    pdfauthor={P. Sopasakis},     % author
    pdfsubject={PID Controller Quiz},   % subject of the document
    pdfcreator={P. Sopasakis},   % creator of the document
    pdfproducer={P. Sopasakis}, % producer of the document
    pdfnewwindow=true,      % links in new PDF window
    colorlinks=true,       % false: boxed links; true: colored links
    linkcolor=red,          % color of internal links (change box color with linkbordercolor)
    citecolor=blue,        % color of links to bibliography
    filecolor=magenta,      % color of file links
    urlcolor=cyan           % color of external links
}


% This is how it should be for the final submission:
\theoremstyle{plain}
\newtheorem{theorem}{Theorem}
\newtheorem{assumption}[theorem]{Assumption}
\newtheorem{lemma}[theorem]{Lemma}
\newtheorem{definition}[theorem]{Definition}
\newtheorem{proposition}[theorem]{Proposition}
\newtheorem{corollary}[theorem]{Corollary}


% CUSTOM COMMANDS
\renewcommand{\Re}{\mathbb{R}}
\newcommand{\uu}{\mathbf{u}}
\newcommand{\N}{\mathbb{N}}
\newcommand{\NN}{\mathcal{N}}
\newcommand{\FF}{\mathfrak{F}}
\newcommand{\expect}{\mathbb{E}}
\newcommand{\prob}{\mathrm{P}}
\newcommand{\sets}{\mathrm{sets}}
\newcommand{\EE}{\mathscr{E}}
\newcommand{\LL}{\mathcal{L}}
\newcommand{\dfn}{\mathrel{:}=}
\newcommand{\bet}{\mathrm{bet}}

\renewcommand{\d}{\mathrm{d}}
\newcommand{\q}{\mathfrak{q}}
\newcommand{\p}{\mathfrak{p}}
\newcommand{\smallmat}[1]{\left[ \begin{smallmatrix}#1 \end{smallmatrix} \right]}

% RENEWED COMMANDS
\renewcommand{\Re}{\mathbb{R}}

%opening
\title[PID Controller]{Quiz: The PID controller}


\author[P. Sopasakis]{Pantelis Sopasakis}


\thanks{Quiz for the PID controller. For questions you may reach me by email at \url{p.sopasakis@gmail.com} 
       or in person in Office No. ... during office hours. Last updated:~\today.}
\begin{document}


\maketitle
Quiz questions
\textbf{}
\begin{enumerate} 
 
 \item The \textbf{input} signal to a \textbf{controller} is 
    \begin{enumerate}
     \item The error, $e(t) = y^{sp}(t) - y^{m}(t)$
     \item The set-point, $y^{sp}(t)$
     \item The measured system variable
     \item The control action
    \end{enumerate}

 \item The \textbf{limit} of the \textbf{error} at infinity is called...
    \begin{enumerate}
     \item Offset
     \item Set-point
     \item Measurement
     \item Feedback
    \end{enumerate}
    
 \item What is the \textbf{main reason} for using the \textbf{integral} in a PID controller?
  \begin{enumerate}
   \item So that the error converges
   \item The controlled variable ultimately returns to its set point after a disturbance 
   \item The derivative of the controlled variable goes to zero
   \item The controlled variable does not oscillate
  \end{enumerate}    
    
 \item Which component of a PID controller is used to \textbf{eliminate offset}?
  \begin{enumerate}
   \item Proportional
   \item Integral
   \item Derivative
   \item Feedback
  \end{enumerate}



  \item A system is controlled with a P controller and it \textbf{oscillates} too much. 
        We should...
  \begin{enumerate}
   \item Increase the proportional gain
   \item Decrease the proportional gain
  \end{enumerate}
  
  
  \item A quadcopter's altitude is controlled with a PD controller and there
        is \textbf{offset}. We should...        
  \begin{enumerate}
   \item Increase the proportional gain
   \item Decrease the proportional gain
   \item Decrease the derivative gain
   \item Introduce an integral mode
  \end{enumerate}
        
  \item An aircraft's roll is controlled with a PID controller, but it \textbf{wobbles}
        too much. We should...
  \begin{enumerate}
   \item Increase the proportional gain
   \item Increase the integral gain
   \item Increase the derivative gain
   \item Decrease the derivative gain
  \end{enumerate}
  
  


  \item In a software implementation of the PID controller, the \textbf{integral} 
        is replaced by
  \begin{enumerate}
   \item sum of errors
   \item successive differences of errors
  \end{enumerate}
        
  \item In a software implementation of the PID controller, the \textbf{derivative} 
        is replaced by
  \begin{enumerate}
   \item sum of errors
   \item successive differences of errors
  \end{enumerate}      
\end{enumerate}

\end{document}