\documentclass[a4paper,10pt,reqno]{amsart}

% \usepackage[utf8x]{inputenc}
% \usepackage[foot]{amsaddr}
% \usepackage{amsmath,amsfonts,amssymb,amsthm}
% \usepackage[margin=0.95in]{geometry}
% \usepackage{color}
% \usepackage[dvipsnames]{xcolor}
% \usepackage{mathtools,mathrsfs}
% 
% \usepackage{hyperref}

\usepackage[utf8]{inputenc}
\usepackage[foot]{amsaddr}
\usepackage{amsmath,amsfonts,amssymb,amsthm,mathrsfs}
\usepackage[margin=0.8in]{geometry}
\usepackage{color}
\usepackage{units}
\usepackage[dvipsnames]{xcolor}

 \input{toc-config.tex}

\usepackage{mathtools}
\usepackage{enumerate}
\usepackage{mathrsfs} 			% for mathscr
\usepackage{hyperref}

\definecolor{CommentGreen}{rgb}{0.0,0.4,0.0}
\definecolor{Background}{rgb}{0.9,1.0,0.85}
\definecolor{lrow}{rgb}{0.914,0.918,0.922}
\definecolor{drow}{rgb}{0.725,0.745,0.769}

\usepackage{listings}
\usepackage{textcomp}
\lstloadlanguages{Matlab}%
\lstset{
    language=Matlab,                        % Use MATLAB
    upquote=true,
    frame=single,                           % Single frame around code
    basicstyle=\small\ttfamily,             % Use small true type font
    backgroundcolor=\color{Background},
    keywordstyle=[1]\color{blue}\bfseries,        % MATLAB functions bold and blue
    keywordstyle=[2]\color{purple},         % MATLAB function arguments purple
    keywordstyle=[3]\color{black}\bfseries,  % User functions underlined and blue
    identifierstyle=,                       % Nothing special about identifiers
                                            % Comments small dark green courier
    commentstyle=\usefont{T1}{pcr}{m}{sl}\color{CommentGreen}\small,
    stringstyle=\color{purple},             % Strings are purple
    showstringspaces=false,                 % Don't put marks in string spaces
    tabsize=5,                              % 5 spaces per tab
    %
    %%% Put standard MATLAB functions not included in the default
    %%% language here
    morekeywords={properties,methods,classdef},
    %
    %%% Put MATLAB function parameters here
    morekeywords=[2]{handle},
    %
    %%% Put user defined functions here
    morecomment=[l][\color{blue}]{...},     % Line continuation (...) like blue comment
    numbers=none,                           % Line numbers on left
    firstnumber=1,                          % Line numbers start with line 1
    numberstyle=\tiny\color{blue},          % Line numbers are blue
    stepnumber=1,                            % Line numbers go in steps of 5,
    xleftmargin=10pt,
    xrightmargin=10pt
}

\lstdefinelanguage{json}{
   upquote=true,
    frame=single,                           % Single frame around code
    basicstyle=\small\ttfamily,             % Use small true type font
    backgroundcolor=\color{Background},
    keywordstyle=[1]\color{blue}\bfseries,        % MATLAB functions bold and blue
    keywordstyle=[2]\color{purple},         % MATLAB function arguments purple
    keywordstyle=[3]\color{black}\bfseries,  % User functions underlined and blue
    identifierstyle=,                       % Nothing special about identifiers
                                            % Comments small dark green courier
    commentstyle=\usefont{T1}{pcr}{m}{sl}\color{CommentGreen}\small,
    stringstyle=\color{purple},             % Strings are purple
    showstringspaces=false,                 % Don't put marks in string spaces
    tabsize=5,                              % 5 spaces per tab
    numbers=none,                           % Line numbers on left
    firstnumber=1,                          % Line numbers start with line 1
    numberstyle=\tiny\color{blue},          % Line numbers are blue
    stepnumber=1,                            % Line numbers go in steps of 5,
    xleftmargin=10pt,
    xrightmargin=10pt
}

\numberwithin{equation}{section}
\synctex=1

\hypersetup{
    bookmarks=true,         % show bookmarks bar?
    unicode=false,          % non-Latin characters in Acrobat’s bookmarks
    pdftoolbar=true,        % show Acrobat’s toolbar?
    pdfmenubar=true,        % show Acrobat’s menu?
    pdffitwindow=false,     % window fit to page when opened
    pdfstartview={FitH},    % fits the width of the page to the window
    pdftitle={PID Homework},    % title
    pdfauthor={P. Sopasakis},     % author
    pdfsubject={Homework},   % subject of the document
    pdfcreator={P. Sopasakis},   % creator of the document
    pdfproducer={P. Sopasakis}, % producer of the document
    pdfnewwindow=true,      % links in new PDF window
    colorlinks=true,       % false: boxed links; true: colored links
    linkcolor=red,          % color of internal links (change box color with linkbordercolor)
    citecolor=blue,        % color of links to bibliography
    filecolor=magenta,      % color of file links
    urlcolor=cyan           % color of external links
}


% This is how it should be for the final submission:
\theoremstyle{plain}
\newtheorem{theorem}{Theorem}
\newtheorem{assumption}[theorem]{Assumption}
\newtheorem{lemma}[theorem]{Lemma}
\newtheorem{definition}[theorem]{Definition}
\newtheorem{proposition}[theorem]{Proposition}
\newtheorem{corollary}[theorem]{Corollary}


% CUSTOM COMMANDS
\renewcommand{\Re}{\mathbb{R}}
\newcommand{\uu}{\mathbf{u}}
\newcommand{\N}{\mathbb{N}}
\newcommand{\NN}{\mathcal{N}}
\newcommand{\FF}{\mathfrak{F}}
\newcommand{\expect}{\mathbb{E}}
\newcommand{\prob}{\mathrm{P}}
\newcommand{\sets}{\mathrm{sets}}
\newcommand{\EE}{\mathscr{E}}
\newcommand{\LL}{\mathcal{L}}
\newcommand{\dfn}{\mathrel{:}=}
\newcommand{\bet}{\mathrm{bet}}

\renewcommand{\d}{\mathrm{d}}
\newcommand{\q}{\mathfrak{q}}
\newcommand{\p}{\mathfrak{p}}
\newcommand{\smallmat}[1]{\left[ \begin{smallmatrix}#1 \end{smallmatrix} \right]}

% RENEWED COMMANDS
\renewcommand{\Re}{\mathbb{R}}

%opening
\title[PID Controller]{Introduction to the PID controller and its Software Implementation --- Homework}


\author[P. Sopasakis]{Pantelis Sopasakis}

% \address[P. Sopasakis]{KU Leuven, Department of Electrical Engineering (ESAT), STADIUS Center for Dynamical Systems, Signal Processing and Data Analytics \& Optimization in Engineering (OPTEC), Kasteelpark Arenberg 10, 3001 Leuven, Belgium. Email addresses: \href{mailto:pantelis.sopasakis@kuleuven.be}{pantelis.sopasakis@kuleuven.be} and 
% \href{mailto:panos.patrinos@esat.kuleuven.be}{panos.patrinos@esat.kuleuven.be}.}
\thanks{Homework for the course ``The PID controller and its software implementation.'' 
        The slides are available at \url{https://alphaville.github.io/qub/pid-101/}. 
        The associated MATLAB source code can be found at \url{https://alphaville.github.io/qub/matlab/}.
        For questions you may reach me by email at 
        \href{mailto:p.sopasakis@gmail.com}{\texttt{p.sopasakis@gmail.com}} 
       or in person in Office No. ... during office hours. Last updated:~\today.}
\begin{document}


\maketitle
\pagestyle{empty}
\pagenumbering{gobble}

\noindent \textbf{Guidelines:} 
Solve the following exercises and send your answers by email to
\href{mailto:p.sopasakis@gmail.com}{\texttt{p.sopasakis@gmail.com}}.
Your answers should be submitted in a single PDF file along with a ZIP file with your source 
code.

\vspace{1em}

\noindent \textbf{Evaluation criteria:} Any correct answer is acceptable. You will be evaluated 
for the correctness of your answers and source code (85\%) and the quality and readability of 
your source code (15\%). The relative importance of each question is indicated in brackets.

\vspace{1em}

\noindent  \textbf{Deadline:} 20 February, 2019, by close of play (17:00)

\vspace{1em}
 In the lecture, we presented a \textit{discrete-time approximation} of the 
       PID controller using the following integral approximation
       \begin{equation}
        \int_{t_{k-1}}^{t_k} e(\tau)\, {}\mathrm{d}\tau {}\approx{} T_s e(t_k)\tag{1a}
       \end{equation}       
       which is known as the \textit{rectangle rule}. We also used the \textit{backward differences}
       approximation for the derivative of the error, that is,
       \[
        \frac{\d e(t_k)}{\d t} \approx \frac{e(t_k) - e(t_{k-1})}{T_s}\tag{1b}\label{eq:1b}.
       \]

      \begin{enumerate}
	\item {} \label{it:discrete-pid-trapezoidal}
	      {}[10\%]{} Give a discrete-time approximation of the PID controller using the 
	      \textit{backward differences} approximation for the derivative of the error (Equation \eqref{eq:1b})
	      and the \textit{trapezoidal rule} --- or \textit{trapezoid rule} --- that is, 
	      \[
	      \int_{t_{k-1}}^{t_k} e(\tau)\, {}\mathrm{d}\tau 
	      {}\approx{} 
	      \frac{T_s}{2} \left(e(t_k) + e(t_{k-1})\right),\tag{2}
	      \]
	      for the approximation of the integral of the error.
	\item {}[10\%]{} Write some simple pseudocode for the discrete-time PID controller
	      you derived in question No. \ref{it:discrete-pid-trapezoidal} using the 
	      trapezoidal rule.
	\item {}\label{it:pid-implementation} {}[15\%]{} Implement the above discrete-time PID controller 
	      in MATLAB. You may build up on the source 
	      code found at \href{https://alphaville.github.io/qub/matlab/}{https://alphaville.github.io/qub/matlab/}{.}
	\item {}[15\%]{} Compare the discrete-time PID controller implementation you 
	      produced in question No. \ref{it:pid-implementation}
	      to the one found at \href{https://alphaville.github.io/qub/matlab/}%
	      {https://alphaville.github.io/qub/matlab/}{.}
	      Try out different sampling times.	According to your observations, how does the sampling time
	      affect the behaviour of the closed-loop system?
	\item {}[20\%]{} Suppose that due to a technical glitch, there is a constant bias of $u_{\mathrm{bias}} = 5^\circ$
	      on the steering angle, that is, the controller commands a control action $u(t)$ to the 
	      wheels, but the actual steering angle is $u(t) + u_{\mathrm{bias}}$.
	      Perform simulations to demonstrate that a PD controller cannot eliminate the offset.
	      According to your observations, what is the effect that $K_p$ and $K_d$ have on the offset? 
        \item \label{it:tuning}
	      {}[20\%]{} 
              Experiment with the values of the proportional gain ($K_p$), the derivative gain ($K_d$)
              and the integral gain ($K_i$) and determine values that lead to a nice, non-oscillatory behaviour with no 
              offset (there is no \textit{single} correct answer). According to your observations, 
              how do $K_p$, $K_d$ and $K_i$ affect the frequency and amplitude of the oscillations 
              of the error?
        \item {}[10\%]{} Simulate the controlled system starting from the initial position $p_x(0)=\unit[0]{m}$ and $p_y(0)=\unit[1]{m}$
              and initial orientation $\theta(0) = \unit[0]{rad}$ towards the set-point $p_y^{\mathrm{sp}} = \unit[1.5]{m}$
              using the PID controller from question No. \ref{it:tuning}.
      \end{enumerate}


\end{document}